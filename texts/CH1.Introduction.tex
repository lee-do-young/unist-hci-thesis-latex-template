\section{Introduction} 

\subsection{reference}
Your introduction should be here. You may need a reference~\cite{Lee:2018}.

\subsection{table}
Table.~\ref{tab:mytable} is a sample table. 

\begin{table}[h]
\centering
\begin{tabular}{ |p{3cm}||p{3cm}|p{3cm}|p{3cm}|  }
 \hline
 \multicolumn{4}{|c|}{Country List} \\
 \hline
 Country Name     or Area Name& ISO ALPHA 2 Code &ISO ALPHA 3 Code&ISO numeric Code\\
 \hline
 Afghanistan   & AF    &AFG&   004\\
 Aland Islands&   AX  & ALA   &248\\
 Albania &AL & ALB&  008\\
 Algeria    &DZ & DZA&  012\\
 American Samoa&   AS  & ASM&016\\
 Andorra& AD  & AND   &020\\
 Angola& AO  & AGO&024\\
 \hline
\end{tabular}
\caption{My table.} \label{tab:mytable}
\end{table}

You can make it simply in https://www.tablesgenerator.com/

\subsection{figure}

Fig.~\ref{fig:myfigure} is a sample figure. 

\begin{figure}[h]
\centering
\includegraphics[width=5in]{figures/myfigure.pdf}
\caption{My figure.} \label{fig:myfigure}
\end{figure}

%%%% use the following in the case of eps figure
%\begin{figure} 
%    \centering
%    \epsfig{file=model.eps, width = 0.9\linewidth}
%    \caption{System model with complete bipartite graph. The maximum weighted matching is marked by circles.}
%    \label{fig:model}
%\end{figure}

\subsection{equation}

equation (\ref{eq:myeq}) can be written as 
%
\begin{equation}
\label{eq:myeq}
\begin{split}
	A 		&= B + C, \\
    D + E	&= F.
\end{split}
\end{equation}

\subsection{subsection}
This is a subsection.

\subsubsection{subsubsection}
This is a subsubsection.

\paragraph{paragraph}
This is a paragraph. (equivalent to subsubsubsection)

\newpage